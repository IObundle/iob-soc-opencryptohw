The IOb-Versat~\cite{iob-versat} CGRA aims to provide a framework and a set of
tools to facilitate the development of CGRA architectures.
A CGRA is comprised of multiple interconnected functional units (FUs). Both the
FUs and the connections between then can be configured by a CPU at runtime to
change the functionality and the dataflow of the hardware architecture.

For a particular application, the developer needs to extablish the required
FUs. The FUs can be choosen from a set of FUs supported by IOb-Versat by
default. These include external memory access, addition, multiplication and
accumulation, internal memory blocks, simple arithmetic and logic unit,
multiplexing among other FUs.

IOb-Versat also supports custom designed FUs in Verilog. The integration with
IOb-Versat requires the development of corresponding FU wrappers in C/C++ to
allow simulation and software driver control.

The computation datapaths are described in C/C++ by interconnecting FU wrappers.
IOb-Versat provides tools to automatically synchronize the data along the
datapaths and to merge multiple computation datapaths.
The datapaths are synchronized using the delays of each FU type and the
datapath graph. 
Multiple computation datapaths can be merged to reduce the hardware footprint
in the case of datapaths that use common sets of FUs.

The accelerated routines in software are replaced by IOb-Versat drivers that
configure the CGRA and execute the datapath. The IOb-Versat accesses the system 
memory directly through a direct memory access (DMA) block, thus bypassing the 
CPU+cache subsystem.
