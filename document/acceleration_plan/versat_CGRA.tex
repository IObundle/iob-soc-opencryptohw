The IOb-Versat~\cite{iob-versat} CGRA aims to provide a framework and a set of
tools to facilitate the development of CGRA architectures. A CGRA is a set of
multiple interconnected functional units (FUs). A CPU can configure the FUs and
the connections between them to change the functionality and data flow of the
hardware architecture at runtime.

For a particular application, the developer needs to establish the required
FUs. The developer can choose from a set of default FUs supported by
IOb-Versat. These include external memory access, addition, multiplication and
accumulation, internal memory blocks, simple arithmetic and logic unit and
multiplexing, among other FUs.

IOb-Versat also supports custom designed FUs in Verilog. The integration with
IOb-Versat requires the development of corresponding FU wrappers in C/C++ to
allow simulation and software driver control.

The computation datapaths are described in C/C++ by interconnecting FU
wrappers. IOb-Versat provides tools to automatically synchronize the data along
the datapaths and merge multiple computation datapaths. The datapaths are
synchronized using the delays of each FU type and the datapath graph. For cases
where different datapaths have sets of FUs in common, the IOb-Versat can merge
the computation datapaths to reduce the hardware footprint.

IOb-Versat drivers that configure the CGRA and execute the datapath replace the
software routines. The IOb-Versat accesses the system memory directly through a
direct memory access (DMA) block, thus bypassing the CPU+cache subsystem.
Figure~\ref{fig:system_versat} presents an IOb-SoC system with Versat CGRA as a
peripheral. The CPU accesses the Versat CGRA configuration values, sends run
commands or reads status. 

\begin{figure}[!htbp]
    \centerline{\includegraphics[width=12cm]{./figures/system_block_diagram.pdf}}
    \vspace{0cm}\caption{Example of Versat CGRA integration in IOb-SoC.}
    \label{fig:system_versat}
\end{figure}

Figure~\ref{fig:versat_bd} provides a block diagram of a Versat CGRA. The CPU 
accesses affect the configuration block. The configuration block fans out the 
configuration values for each specific FU. Versat supports any topology between
the FUs in the configured datapaths. The \textbf{vRead} and \textbf{vWrite} are
specified in the block diagram to highlight the interface with external memory.

\begin{figure}[!htbp]
    \centerline{\includegraphics[width=12cm]{./figures/versat_example.pdf}}
    \vspace{0cm}\caption{Example of Versat CGRA block diagram.}
    \label{fig:versat_bd}
\end{figure}

