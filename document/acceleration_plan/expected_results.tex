The best-case performance for the acceleration of a hash function is to compute
the working variables equations~(\ref{eq:hash_iter}) once per cycle. This case
gives 64 clock cycles to hash a message block. 
The test messages used in the profiling application have increasing sizes from
0 to 512 bits with a step size of 8 bits. The test messages with sizes from 0
to 440 bits are split into one message block. The remaining test messages
require two message blocks. 
The minimum number of clock cycles to execute the hashing of message blocks for
the complete test is given in (\ref{eq:hash_cycles_test}). The test contains 65
messages with the last 9 split into two message blocks during hashing.

\begin{align}
    \begin{split}
    hash\_cycles &= 64 \times ( 65 + 9 ) = 4736 \ clock \ cycles \\
    4736 \ clock \ cycles &\Leftrightarrow \frac{4736}{100\times 10^{6}}\times 10^{6} = 47 \ \mu s.
    \end{split}
\label{eq:hash_cycles_test}
\end{align}

The acceleration proposed in section~\ref{sec:accel_proposal} replaces the 
\texttt{sha\_init()}, \texttt{sha\_finalize()} and \texttt{sha\_ctxrelease()} 
functions. From table~\ref{tab:prof1}, these functions take a total of 
$ (886 + 22816 + 571) = 24273 \ \mu s$ to execute. Assuming that the SHA-256
acceleration takes the time calculated in (\ref{eq:hash_cycles_test}), the
expected speedup is given by:

\begin{equation}
    \frac{Total\ time}{Accel\ time} = \frac{28381}{28381 - 24273 + 47} \approx 6.83.
\label{eq:expected_speedup}
\end{equation}
