A secure and trustworthy internet needs computationally intensive cryptography
algorithms. For security, performance and applicability to multiple platforms,
it is best if these algorithms are executed in hardware with some of the
flexibility of software.

The OpenCryptoHW project, funded by the NGI Assure program, proposed the use of
Coarse Grained Reconfigurable Arrays (CGRAs)~\cite{CGRA:Overview} to accomplish
this objective, and chose the SHA256 and the AES256 algorithms to illustrate
the ideas proposed.

The project's assumptions are the following: internet devices use a main
Central Processing Unit and an Ethernet Medium Access Controller (MAC) core.
Each internet device is engaged in frequent secure network communications. The
software and hardware descriptions must be open-source to ensure trust. 
OpenCryptoHW uses the VexRiscV processor~\cite{VexRiscv} and the IOb-Eth MAC
core~\cite{iob-eth} to accomplish these objectives. Moreover, it uses the
IOb-Versat CGRA~\cite{iob-versat} to accelerate the cryptography algorithms.

This document describes the SHA256~\cite{SHA_NIST_FIPS} acceleration plan
running on the IOb-SoC-SHA system~\cite{iob-soc-opencryptohw}, according to Milestone 7
of the OpenCryptoHW project. The document has five parts:
\begin{itemize}
\item the first part provides a brief introduction to the SHA-256 algorithm.
\item the second part introduces the IOb-Versat framework and related tools.
\item the third part presents profiling data for the SHA256 application
    running exclusively on the RISC-V CPU.
\item the fourth part elaborates on the profiling conclusions and establishes
    functional unit architectures to accelerate the application.
\item the fifth part outlines a prediction for the expected results from
    implementing the acceleration strategy.
\end{itemize}









