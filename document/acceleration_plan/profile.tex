The execution time for the SHA256 implementation \cite{iob-soc-sha} is
presented in Table~\ref{tab:prof1}. The program runs exclusively on software
with instructions stored in internal memory and data stored in external memory
(DDR). The system uses the VexRiscv CPU \cite{iob-vexriscv} at a clock
frequency of 100 MHz.

\begin{table}[h]
    \centering
    \input{profile_tab.tex}
    \caption{Baseline application profile data.}
    \label{tab:prof1}
\end{table}

The profile analysis tracks the time in clock cycles since the input data is in
external memory until the CPU writes the output data to external memory.

The results from Table~\ref{tab:prof1} demonstrate that about 80\% of the
execution time is used to run the \texttt{sha\_finalize()} function, in
particular, the \texttt{crypto\_hashblocks()} function. The functions and macro
calls inside the \texttt{crypto\_hashblocks()} function have the same order of
magnitude with regards to duration.

The acceleration efforts should be focussed on the
\texttt{crypto\_hashblocks()} function and respective subfunction and macro
calls.
