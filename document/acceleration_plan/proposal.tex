Figure~\ref{fig:hash_flowchart} presents the flowchart of the
\texttt{crypto\_hashblocks()} function.
The function starts by reading the current state from memory. Then each message
block of 64 bytes (256 bits) of the input data is used to hash the message
block.
After all input data is used, the new state is written to memory.
The majority of computations take place inside the loop to hash the message
blocks.

\begin{figure}[!htbp]
    \centerline{\includegraphics[width=14cm]{./figures/crypto_hashblock_flowchart.pdf}}
    \vspace{0cm}\caption{\texttt{crypto\_hashblocks()} function flowchart.}
    \label{fig:hash_flowchart}
\end{figure}

Figure~\ref{fig:main_hashblock_diagram} presents a block diagram for the
process of hashing a message block.
The message block hashing output is the accumulation from the initial state
with the new computed state.
The new computed state is the output of the sequence of \textbf{F} blocks. Each
\textbf{F} block receives three inputs: a set of constants stored in the
\textbf{cMem} blocks; the previous or initial state in the \textbf{a-h}
variables; and a set of words from the message scheduling array \textbf{w}.

\begin{figure}[!htbp]
    \centerline{\includegraphics[width=18cm]{./figures/main_hashblock_diagram.pdf}}
    \vspace{0cm}\caption{Hash message block block diagram.}
    \label{fig:main_hashblock_diagram}
\end{figure}

Each set of 16 \textbf{w} words is obtained from the input data or by applying a
previous set of words to the \textbf{M} block.

The proposed accelerator architecture has 5 functional unit (FU) types:
\begin{itemize}
    \item 1 Vread to store the input data;
    \item 1 State FU to store and accumulate the \textbf{a-h} state variables;
    \item 3 Memories to store the constants (equivalent to \textbf{cMem}
        blocks);
    \item 3 \textbf{M} FUs that generate a new set of message schedule array
        words;
    \item 4 \textbf{F} FUs that perform the compression function.
\end{itemize}

The Vread, state and memory FUs are default FUs from Versat. The \textbf{M} and
\textbf{F} FUs are custom units built specifically for the SHA256 application.

The Vread FU reads data from any memory address in the system and provides input
data to other FUs.

The memory FU is an auxiliar memory that holds constant values used as input to
other FUs.

The state FU is a set of accumulation registers. The registers can be
initialized with a value by the CPU. The register values can be used as input or
output of other FUs.

The \textbf{M} FU performs the logic equivalent to the \texttt{EXPAND\_32}
macro defined in the \texttt{sha.c} source code. The macro generates 16 new
words for the message schedule array. Each new word is generated by applying
logic operations to previous words.

The \textbf{F} FU performs the logic equivalent to a group of 16
\texttt{F\_32(w,k)} macros defined in the \texttt{sha.c} source code.
Each \texttt{F\_32(w,k)} macro updates the state values (\textbf{a} to
\textbf{h}) using one message schedule word, one constant value and the
previous state values.

