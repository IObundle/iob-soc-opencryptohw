The SHA-256 algorithm \cite{SHA_NIST_FIPS} is a secure hash algorithm that
receives input data of any size up to $2^{64}$ bits and computes an output of
256 bits. This output is called a message digest.
The SHA-256 algorithm is divided into two main stages: preprocessing and hash
computation.

\subsection{Preprocessing}
\label{subsec:preprocessing}
The preprocessing stage pads the input data to obtain an input size multiple of
512 bits. Given a message $M$ of size $\lambda$ bits. The padding process
appends the bit "1" to the end of the message followed by $\delta$ "0" bits
such that $\delta$ is the smallest positive integer that
solves~(\ref{eq:sha_padding}).

\begin{equation}
\lambda + 1 + \delta \equiv 448 \mod 512.
\label{eq:sha_padding}
\end{equation}

After the padded zeroes, the message is appended with the 64 bit representation
of the size of the original message $\lambda$. At the end of this process the 
padded message size is a multiple of 512 bits. The padded input is divided into
blocks of 512 bits which can be represented as sixteen words of 32 bit. 

The preprocessing stage also sets the initial state for the hash value. The hash
state values are a set of eight 32 bit words. For the SHA-256 algorithm, the
initial values are the first 32 bits of the fractionary part of the square root
of the first eight prime numbers.

\subsection{Hash Computation}
\label{subsec:hash_computation}

Figure~\ref{fig:hash_flowchart} presents the hash computation stage which
processes one message block at a time. For each iteration $i$, the hash stage
values $H_{0}^{(i+1)}, H_{1}^{(i+1)},..., H_{7}^{(i+1)}$ are updated using a
message schedule of sixty-four 32 bit words $W_0, W_1,..., W_{63}$, the previous hash
state values $H_{0}^{(i)}, H_{1}^{(i)},..., H_{7}^{(i)}$ and 64 constants 
$K_{0},K_{1},...,K_{63}$ of 32 bit each.

\begin{figure}[!htbp]
    \centerline{\includegraphics[width=14cm]{./figures/crypto_hashblock_flowchart.pdf}}
    \vspace{0cm}\caption{SHA-256 hash function flowchart.}
    \label{fig:hash_flowchart}
\end{figure}

The initial hash state values and the message block words are obtained from the 
preprocessing stage as described in section~\ref{subsec:preprocessing}.
The 64 constants $K_{t}$ are the 32 fractionary bits of the cubic roots of the
first 64 prime numbers.

The sixty-four message schedule words are the sixteen 32 bit words from the
input message block plus 48 generated words. Each generated word $W_{t}$ is
computed by the operations presented in~(\ref{eq:gen_w}). The addition is
modulo $2^{32}$.

\begin{equation}
    W_{t} = \sigma_{1}(W_{t-2}) + W_{t-7} + \sigma_{0}(W_{t-15}) + W_{t-16}, \ \ \ 16 \leq t \leq 63.
\label{eq:gen_w}
\end{equation}

Where $\sigma_{0}()$ and $\sigma_{1}()$ functions are a set of logic operations
defined in~(\ref{eq:sigma_funcs}). $ROTR^{n}(x)$ is a rotate right $n$ bits
function and $SHR^{n}(x)$ is a right shift $n$ bits operation.

\begin{align}
    \begin{split}
        \sigma_{0}(x) &= ROTR^{7}(x) \oplus ROTR^{18}(x) \oplus SHR^{3}(x), \\
        \sigma_{1}(x) &= ROTR^{17}(x) \oplus ROTR^{19}(x) \oplus SHR^{10}(x).
    \end{split}
\label{eq:sigma_funcs}
\end{align}

The initial hash state values initialize a set of working variables $a, b, c,
d, e, f, g, h$:

\begin{align}
    \begin{split}
        a &= H_{0}^{(i-1)} \\
        b &= H_{1}^{(i-1)} \\
        c &= H_{2}^{(i-1)} \\
        d &= H_{3}^{(i-1)} \\
        e &= H_{4}^{(i-1)} \\
        f &= H_{5}^{(i-1)} \\
        g &= H_{6}^{(i-1)} \\
        h &= H_{7}^{(i-1)}.
    \end{split}
\label{eq:init_state}
\end{align}

The working variables are updated for 64 iterations ($ 0 \leq t \leq 63$),
following the algorithm in~(\ref{eq:hash_iter}). The functions $\Sigma_1(x)$,
$Ch(x,y,z)$, $\Sigma_0(x)$ and $Maj(x,y,z)$ are defined
in~(\ref{eq:Sigma_Ch_Maj}). $Ch(x,y,z)$ is a choice operation: if $x$ is 1, the
output is $z$, otherwise outputs $y$. $Maj(x,y,z)$ outputs the most common
value between the three inputs.

\begin{align}
    \begin{split}
        T_1 &= h + \Sigma_1(e) + Ch(e,f,g) + K_{t} + W_{t} \\
        T_2 &= h + \Sigma_0(a) + Maj(a,b,c) \\
        h &= g \\
        g &= f \\
        f &= e \\
        e &= d + T_1 \\
        d &= c \\
        c &= b \\
        b &= a \\
        a &= T_1 + T_2.
    \end{split}
\label{eq:hash_iter}
\end{align}

\begin{align}
    \begin{split}
        \Sigma_{0}(x) &= ROTR^{2}(x) \oplus ROTR^{13}(x) \oplus ROTR^{22}(x), \\
        \Sigma_{1}(x) &= ROTR^{6}(x) \oplus ROTR^{11}(x) \oplus ROTR^{25}(x), \\
        Ch(x,y,z) &= (x \land y) \oplus ( \neg x \land z), \\
        Maj(x,y,z) &= (x \land y) \oplus (x \land z) \oplus (y \land z).
    \end{split}
\label{eq:Sigma_Ch_Maj}
\end{align}

The results $H_{0}^{(i+1)}, H_{1}^{(i+1)},..., H_{7}^{(i+1)}$ of hashing
iteration $i$ are the final values of the working variables $a,..., h$.
The resulting hash state values of one iteration are used as the input of the
next iteration until all message blocks have been used. The message digest is
the concatenation of the hash state values at the final iteration.

